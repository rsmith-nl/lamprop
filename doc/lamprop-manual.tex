% file: lamprop-manual.tex
% vim:fileencoding=utf-8:fdm=marker:ft=tex
%
% Copyright © 2018 R.F. Smith <rsmith@xs4all.nl>. All rights reserved.
% Created: 2018-05-13 23:48:04 +0200
% Last modified: 2018-12-02T01:25:10+0100

\newcommand{\twodigits}[1]{\ifnum#1<10 0\number#1\else\number#1\fi}
\newcommand{\TheDate}{\number\year-\twodigits\month-\twodigits\day}
\newcommand{\TheTitle}{Lamprop manual}

\documentclass[a4paper,landscape,oneside,11pt,twocolumn]{memoir}
\usepackage{fontspec}
\usepackage{graphicx}
\usepackage{siunitx}
\usepackage{ifthen}
\usepackage{enumitem}
\usepackage{listings}

% Set fonts for XeTeX {{{1
\setmainfont{Alegreya}[
    SmallCapsFont={Alegreya SC},
    SmallCapsFeatures={Letters=SmallCaps},
]
\setsansfont{TeX Gyre Heros}
\setmonofont{TeX Gyre Cursor}
\usepackage[italic]{mathastext}

% Hyperref moet als laatste geladen worden,
\usepackage[bookmarks,pdfborder={0 0 0}]{hyperref}
% met uitzondering van uitbreidingen erop.
\usepackage{memhfixc}

% Grootte van de pagina
\setlength{\trimtop}{0pt}
\setlength{\trimedge}{\stockwidth}
\addtolength{\trimedge}{-\paperwidth}
% Een A4 op zijn kant is 297 mm breed en 210 mm hoog.
% We halen 30 mm van de hoogte af, en 20 mm van de breedte.
\settypeblocksize{180mm}{277mm}{*}
% Volgens het log-bestand:
%   Text height and width: 514.20023pt by 788pt
%   Columnsep and columnseprule: 10pt and 0pt
% Dat betekent dat een kolom (788 - 10)/2 = 389pt = 137.2 mm is.
% Een foto op 300 dpi moet dan 389/72*300 = 1621 pixels breed zijn.
% Maar op de breedte van 1200 pixels passen er twee foto's in één
% kolom.

\setulmargins{*}{*}{1}
\setlrmargins{*}{*}{1}
\setheadfoot{\onelineskip}{1.5\onelineskip}
\setheaderspaces{*}{*}{1}
\checkandfixthelayout

% Instellingen van het siunitx pakket.
\sisetup{detect-all=true, mode=text, group-digits=true,
  input-decimal-markers={.,}, output-decimal-marker={.},
  exponent-product=\times, separate-uncertainty=true,
  load-configurations=abbreviations}

% voor enumitem
\setlist[itemize,1]{leftmargin=*}
\setlist[enumerate,1]{leftmargin=*}
\setlist[description,1]{leftmargin=*}
\setlist{noitemsep}

% Instellingen voor polyglossia
%\setmainlanguage[babelshorthands=true]{english}

% Opmaak van  listings {{{1
\lstset{
  language=python,
%  backgroundcolor=\color{inputbackground},
  extendedchars=\true,
  aboveskip=\smallskipamount,
  belowskip=\smallskipamount,
  breaklines=true,
%  basicstyle=\small \ttfamily,
  basicstyle=\small,
  showstringspaces=false,
  commentstyle=\itshape,
  stringstyle=\ttfamily,
  upquote=true,
  columns=fullflexible, % tighter character kerning, like verb
  inputencoding=utf8,
  extendedchars=true,
  literate=
  {á}{{\'a}}1 {é}{{\'e}}1 {í}{{\'i}}1 {ó}{{\'o}}1 {ú}{{\'u}}1
  {Á}{{\'A}}1 {É}{{\'E}}1 {Í}{{\'I}}1 {Ó}{{\'O}}1 {Ú}{{\'U}}1
  {à}{{\`a}}1 {è}{{\`e}}1 {ì}{{\`i}}1 {ò}{{\`o}}1 {ù}{{\`u}}1
  {À}{{\`A}}1 {È}{{\'E}}1 {Ì}{{\`I}}1 {Ò}{{\`O}}1 {Ù}{{\`U}}1
  {ä}{{\"a}}1 {ë}{{\"e}}1 {ï}{{\"i}}1 {ö}{{\"o}}1 {ü}{{\"u}}1
  {Ä}{{\"A}}1 {Ë}{{\"E}}1 {Ï}{{\"I}}1 {Ö}{{\"O}}1 {Ü}{{\"U}}1
  {â}{{\^a}}1 {ê}{{\^e}}1 {î}{{\^i}}1 {ô}{{\^o}}1 {û}{{\^u}}1
  {Â}{{\^A}}1 {Ê}{{\^E}}1 {Î}{{\^I}}1 {Ô}{{\^O}}1 {Û}{{\^U}}1
  {œ}{{\oe}}1 {Œ}{{\OE}}1 {æ}{{\ae}}1 {Æ}{{\AE}}1 {ß}{{\ss}}1
  {ű}{{\H{u}}}1 {Ű}{{\H{U}}}1 {ő}{{\H{o}}}1 {Ő}{{\H{O}}}1
  {ç}{{\c c}}1 {Ç}{{\c C}}1 {ø}{{\o}}1 {å}{{\r a}}1 {Å}{{\r A}}1
  {€}{{\euro}}1 {£}{{\pounds}}1 {«}{{\guillemotleft}}1
  {»}{{\guillemotright}}1 {ñ}{{\~n}}1 {Ñ}{{\~N}}1 {¿}{{?`}}1
}
\lstdefinestyle{plain}{
    basicstyle=\tiny\ttfamily,
    columns=fixed,
    language={},
    backgroundcolor={},
    identifierstyle={}
}

% Definities van locale modificaties. MOET achter de invoeging van het ifthen
% pakket komen!
%\ifthenelse{\VCModified>0}%
%{\newcommand{\locmod}{\textcolor{red}{~(m)}}}{\newcommand{\locmod}{}}


% Document parameters {{{1
\title{\TheTitle}
\author{Roland F. Smith}
\date{\TheDate}

% Kop- en voetteksten
\makeevenhead{plain}{\thetitle}{}{}
\makeoddhead{plain}{\thetitle}{}{}
\makeheadrule{plain}{\textwidth}{\normalrulethickness}
\makefootrule{plain}{\textwidth}{\normalrulethickness}{0pt}
\makeevenfoot{plain}{\theauthor}{\thepage}{\TheDate}
\makeoddfoot{plain}{\theauthor}{\thepage}{\TheDate}

\makepagestyle{logboek}
\makeevenhead{logboek}{\thetitle}{}{\rightmark}
\makeoddhead{logboek}{\thetitle}{}{\rightmark}
\makeheadrule{logboek}{\textwidth}{\normalrulethickness}
\makefootrule{logboek}{\textwidth}{\normalrulethickness}{0pt}
\makeevenfoot{logboek}{\theauthor}{\thepage}{\TheDate}
\makeoddfoot{logboek}{\theauthor}{\thepage}{\TheDate}

% Ruimte voor nummering in lijsten
\cftsetindents{section}{1.5em}{3.0em}
\cftsetindents{figure}{1.5em}{3.0em}
\cftsetindents{table}{1.5em}{3.0em}

\sloppy
\makeindex

% Document info. {{{1
\special{pdf:docinfo <<
/Title (\TheTitle)
/Author (\theauthor)
/Subject (lamprop)
/Keywords (lamprop, manual, Roland Smith)
/CreationDate (D:20180108164304+0100)
>>}

%%%%%%%%%%%%%%%%%%%%%%%%% start van het document %%%%%%%%%%%%%%%%%%%%% {{{1
\begin{document}
% Specifiek voor MEMOIR
% Maak de lijsten minder open.
\tightlists
% Opmaak voor verklarende tekst bij figuren
\hangcaption
\captiontitlefont{\small}
\chapterstyle{section}
% Gebruik twee kolommen in TOC etc.
\doccoltocetc

% Geen inspringen, maar ruimte tussen paragrafen
%\setlength{\parskip}{\baselineskip}
\nonzeroparskip
\setlength{\parindent}{0pt}
\setbeforesecskip{5pt}
\setaftersecskip{1pt}
\setbeforesubsecskip{5pt}
\setaftersubsecskip{1pt}

\begin{titlingpage}
  \setlength{\parindent}{0pt} % Anders klopt uitlijning v.d. \rule's niet.
  \vspace*{\stretch{1}}
  \rule{\linewidth}{1mm}\vspace{5pt}
  \begin{flushright}
      {\Huge \TheTitle}\\[5mm]
      {\huge Roland F. Smith}
  \end{flushright}
  \rule{\linewidth}{1mm}
  \vspace*{\stretch{3}}
  \begin{center}
    \Large Eindhoven 2018
  \end{center}
\end{titlingpage}

% Om paginanummering gelijk te krijgen aan pagina's in PDF.
\setcounter{page}{2}
% Gebruik de standaard opmaak.
\pagestyle{logboek}

% Inhoudsopgave
\begin{KeepFromToc}
\tableofcontents
\end{KeepFromToc}
% Lijst met afbeeldingen.
%\listoffigures
% Lijst met tabellen
%\listoftables
% Index
%\printindex
\clearpage

% Alleen hoofdstuk- en sectie nummers.
\setcounter{secnumdepth}{1}

%%%%%%%%%%%%%%%%%%%% Inleiding %%%%%%%%%%%%%%%%%%%%
\chapter{Introduction} % {{{1

The purpose of this program is to calculate some properties of
fiber-reinforced composite laminates. It calculates:
\begin{itemize}
    \item engineering properties like $E_x$, $E_y$ and $G_{xy}$
    \item thermal properties like $\alpha_x$ and $\alpha_x$
    \item physical properties like density and laminate thickness
    \item stiffness and compliance matrices (\textsc{abd}  and abd)
\end{itemize}


Although these properties are not very difficult to calculate, (the relevant
equations and formulas can be readily found in the available composite
literature) the calculation is time-consuming and error-prone when done by
hand.

This program can \emph{not} calculate the strength of composite laminates;
because there are many different failure modes, strengths of composite
laminates cannot readily be calculated from the strengths of the separate
materials that form the laminate. These strengths have to be determined from
tests.

The original version of this program was written in C, since implementing
it in a spreadsheet proved cumbersome, inflexible and even produced
incorrect results. The C version ran up to 1.3.x.

As an exercise in learning the language, the author ported the program to
the Python programming language. This proved to be a much cleaner, more
maintainable and shorter implementation.

In the meantime, the program was ported from python version 2 to python
version 3 and the core objects (in the \texttt{types.py} file) were made
immutable. Also the output method was made generic to enable output in
different formats, such as \LaTeX, \textsc{html} and \textsc{rtf}.

Additionally, the generally hard to obtain transverse fiber properties
were replaced with properties derived from the matrix.

\chapter{Building and installing the program} % {{{1

\section{Requirements} % {{{2

The main requirements are \texttt{python} (version 3.4 or later) and the
\texttt{numpy} library (version 1.6 or later). Currently the development is
done using \texttt{python} 3.7 and \texttt{numpy} 1.15.

For developers: You will need py.test\footnote{\url{https://docs.pytest.org/}}
to run the provided tests. Code checks are done using
pylama\footnote{\url{http://pylama.readthedocs.io/en/latest/}}. Both should be
invoked from the root directory of the repository.

There are basically two versions of this program; a console version (installed
as \texttt{lamprop}) primarily meant for \textsc{posix} operating systems and
a \textsc{gui} version (installed as \texttt{lamprop-gui}) primarily meant for
ms-windows.

You can try both versions without installing them first, with the following
invocations in a shell from the root directory of the repository.

Use \texttt{python3 -m lamprop.console -h} for the console version, and
\texttt{python3 -m lamprop.gui} for the \textsc{gui} version.

Note that if lamprop is already installed, using the above commands will use
the \emph{installed} version of the lamprop module.


\section{Installation} % {{{2

\begin{itemize}
    \item Unpack the tarball or zipfile, or clone the github repository.
    \item Open a terminal window or (on ms-windows a \texttt{cmd} window).
    \item Change into the lamprop directory.
    \item Run \texttt{python3 setup.py install}. This will install both the module and
the scripts that use it.
\end{itemize}


\chapter{Using the program} % {{{1

There are basically three ways to use lamprop.

\begin{enumerate}
    \item Use the command-line front-end \texttt{lamprop}.
    \item Use the \textsc{gui}-frontend \texttt{lamprop-gui}.
    \item Use the lamprop module directly from Python 3.
\end{enumerate}

The first and second method depend on files written in a domain-specific
language.

\section{The lamprop file format} % {{{2

The file format is very simple. Functional lines have either \texttt{f},
\texttt{r}, \texttt{t}, \texttt{m}, \texttt{l} or \texttt{s} as the first
non-whitespace character. This character must immediately be followed by
a colon \texttt{:}. All other lines are seen as comments and disregarded.

This program assumes specific metric units. The units used below are important
because the program internally calculates the thickness of layers (in mm)
based on the volume fractions and densities of the fibers and resins.

The \texttt{f:}-line line contains a definition of a fiber. The parser
converts this into an instance of a \texttt{Fiber} object The line must
contain the following values, separated by whitespace:
\begin{description}
    \item[$E_1$] Young's modulus in the fiber direction in \si{MPa}.
    \item[$\nu_{12}$] Poisson's constant (dimensionless).
    \item[$\alpha_1$] Coefficient of Thermal Expansion in the fiber direction
        in \si{K^{-1}}.
    \item[$\rho$] Density of the fiber in \si{g/cm^3}.
    \item[$name$] The identifier for the resin. This should be unique in all
        the files read. Contrary to the previous values, this may contain
        whitespace.
\end{description}

Usually, $E_1$ and other properties in the fibre length direction are easily
obtained from a fiber supplier. Previous versions of this program also
required some properties perpendicular to the fiber to calculate transverse
properties of the lamina. Since these values are very hard to obtain, they
have been replaced by values derived from the matrix, according to Tsai(1992).

In the \texttt{tools} subdirectory of the source distribution you will find
a script called \texttt{convert-lamprop.py} to convert old-style fiber lines
to the new format.

The \texttt{r:}-line line contains a definition of a resin. Like with the
fibers, this becomes an instance of a \texttt{Resin} object in the code. The
resin line must contain the following values, separated by whitespace.
\begin{description}
    \item[$E$] Young's modulus in \si{MPa}.
    \item[$\nu$] Poisson's constant (dimensionless).
    \item[$\alpha$] Coefficient of Thermal Expansion in \si{K^{-1}}.
    \item[$\rho$] Density of the resin in \si{g/cm^3}.
    \item[$name$] The identifier for the resin. This should be unique in all
        the files read. Contrary to the previous values, this may contain
        whitespace.
\end{description}

The \texttt{t:} line starts a new laminate. It only contains the name which
identifies the laminate. This name must be unique within the current input
files. It may contain spaces.

The \texttt{m:} line chooses a resin for the laminate. It must appear after
a \texttt{t:} line, and before the \texttt{l:} lines. It must contain the
following values, separated by whitespace:
\begin{description}
    \item[$vf$] The fiber volume fraction. This should be a number between
        0 and 1 or between 1 up to and including 100. In the latter case it
        is interpreted als a percentage.
    \item[$name$] The name of the resin to use. This must have been previously
        declared with an \texttt{r:}-line.
\end{description}

The \texttt{l:} line defines a single layer (lamina) in the laminate. It must be
preceded by a \texttt{t:} and a \texttt{m:} line. It must contain the following values,
separated by whitespace (optional items in brackets):
\begin{description}
    \item[$weight$] The area weight in \si{g/m^2} of the dry fibers.
    \item[$angle$] The angle upwards from the x-axis under which the fibers are oriented.
    \item[($vf$)] Optionally the layer can have a different fiber volume fraction.
    \item[$name$] The name of the fiber used in this layer. This fiber must have been
        declared previously with an \texttt{f:} line.
\end{description}

The last line in a laminate definition can be an \texttt{s:} line, which stands
for "symmetry". This signifies that all the lamina before it are to be added
again in reverse order, making a symmetric laminate stack. An \texttt{s:} line in any
other position is an error.

\begin{lstlisting}[style=plain]
Fiber definition
   E1     v12  alpha1   rho  naam
f: 233000 0.2  -0.54e-6 1.76 Hyer's carbon fiber

Matrix definition
   Em   v    alpha   rho name
r: 4620 0.36 41.4e-6 1.1  Hyer's resin

t: [0/90]s laminate
This is a standard symmetric cross-ply laminate. It has fine extensional
moduli in the fiber directions, but a very low shear modulus.
m: 0.5 Hyer's resin
l: 100  0 Hyer's carbon fiber
l: 100 90 Hyer's carbon fiber
s:
\end{lstlisting}

\section{Material data} % {{{2

Over the years, the author has gathered a lot of data for different fibers
from datasheets provided by the manufacturers. Data for different carbon
fibers is given in \tref{tb:fibers}. In case the $\nu_{12}$ is not
known for a carbon fiber, it is estimated at 0.25. Similarly, if the
$\alpha_1$ is not known, it is estimated at \SI{-0.12e-6}{K^{-1}}. For glass
fibers, $\nu_{12}$ is estimated 0.33 unless known and $\alpha_1$ is estimated
\SI{5e-6}{K^{-1}} unless known.

\begin{table}[!htbp]
  \centering
  \caption{\label{tb:fibers}fibers}
  \begin{tabular}{lrrrrl}% l,c,r
      Name & $E_1$ & $\nu_{12}$ & $\alpha_1$ & $\rho$ & Type\\
      & [\si{MPa}] & [-] & [\si{K^{-1}}] & [\si{g/cm^3}]\\
    \midrule
      Tenax HTA & 238000 & 0.25 & -0.1e-6 & 1.76 & carbon\\
Tenax HTS & 240000 & 0.25 & -0.1e-6 & 1.77 & carbon\\
Tenax STS40 & 240000 & 0.25 & -0.12e-6 & 1.78 & carbon\\
Toracya T300 & 230000 & 0.27 & -0.41e-6 & 1.76 & carbon\\
Torayca T700SC & 230000 & 0.27 & -0.38e-6 & 1.80 & carbon\\
pyrofil TR30S & 235000 & 0.25 & -0.5e-6 & 1.79 & carbon\\
sigrafil CT24-5.0-270/E100 & 270000 & 0.25 & -0.12e-6 & 1.79 & carbon\\
K63712 & 640000 & 0.234 & -1.47e-6 & 2.12 & carbon\\
K63A12 & 790000 & 0.23 & -1.2e-6 & 2.15 & carbon\\
Torayca T800S & 294000 & 0.27 & -0.60e-6 & 1.76 & carbon\\
K13C2U & 900000 & 0.234 & -1.47e-6 & 2.20 & carbon\\
M35J & 339000 & 0.27 & -0.73e-6 & 1.75 & carbon\\
M46J & 436000 & 0.234 & -0.9e-6 & 1.84 & carbon\\
PX35UD & 242000 & 0.27 & -0.6e-6 & 1.81 & carbon\\
Granoc XN-80-60S & 780000 & 0.27 & -1.5e-6 & 2.17 & carbon\\
Granoc XN-90-60S & 860000 & 0.27 & -1.5e-6 & 2.19 & carbon\\
e-glass & 73000 & 0.33 & 5.3e-6 & 2.60 & glass\\
ecr-glass & 81000 & 0.33 & 5e-6 & 2.62 & glass\\
  \end{tabular}
\end{table}

Several resins are shown in \tref{tb:resins}. For resins, $\nu$ is estimated
0.36 unless known and $\alpha$ is estimated \SI{40e-6}{K^{-1}} unless known.

\begin{table}[!htbp]
  \centering
  \caption{\label{tb:resins}Resins}
  \begin{tabular}{lrrrrl}% l,c,r
      Name & $E$ & $\nu$ & $\alpha$ & $\rho$ & Type\\
      & [\si{MPa}] & [-] & [\si{K^{-1}}] & [\si{g/cm^3}]\\
    \midrule
      Epikote EPR04908 & 2900 & 0.25 & 40e-6 & 1.15 & epoxy\\
      Palatal P4-01 & 4300 & 0.36 & 40e-6 & 1.19 & polyester\\
      Synolite 2155-N-1 & 4000 & 0.36 & 40e-6 & 1.22 & polyester\\
      Distitron 3501LS1 & 4100 & 0.36 & 40e-6 & 1.2 & polyester\\
      Synolite 1967-G-6 & 3800 & 0.36 & 40e-6 & 1.165 & \textsc{dcpd}\\
      atlac 430 & 3600 & 0.36 & 55e-6 & 1.145 & vynilester\\
  \end{tabular}
\end{table}

\section{Using the command-line front-end} % {{{2

The command \texttt{lamrop -h} produces the following overview of the options.

\begin{lstlisting}[style=plain]
usage: lamprop [-h] [-l | -H | -r] [-e | -m] [-L | -v]
               [--log {debug,info,warning,error}]
               [file [file ...]]

Calculate the elastic properties of a fibrous composite laminate.

positional arguments:
  file                  one or more files to process

optional arguments:
  -h, --help            show this help message and exit
  -l, --latex           generate LaTeX output (the default is plain text)
  -H, --html            generate HTML output
  -r, --rtf             generate Rich Text Format output
  -e, --eng             output only the layers and engineering properties
  -m, --mat             output only the ABD and abd matrices
  -L, --license         print the license
  -v, --version         show program's version number and exit
  --log {debug,info,warning,error}
                        logging level (defaults to 'warning')
\end{lstlisting}

\section{Using the \textsc{gui} front-end} % {{{2

The \textsc{gui} front-end was written (using \texttt{tkinter}) primarily for
users of ms-windows, since they are generally not used to the command-line
interface. The contents of its window are shown in \fref{fig:lamprop-gui}. The
image shows the looks of the widgets on \textsc{unix}-like operating systems.
On ms-windows follow the native look.

\begin{figure}[!htbp]
  \centerline{\includegraphics[scale=1]{lamprop-gui.png}}
  \caption{\label{fig:lamprop-gui}lamprop \textsc{gui}}
\end{figure}

The \textsf{File} button allows you to load a lamprop file. If a file is
loaded its name is shown right of the button. The \textsf{Reload} button
re-loads a file. The checkboxes below determine which results are shown. If
a file contains different laminates, the dropbox allows you to select
a laminate to display. The textbox at the bottom shows the lamprop output as
text.

Pressing the \texttt{q}-key terminates the program.

\section{Using the \texttt{lamprop} module from Python 3} % {{{2

An example reproducing the results from \fref{fig:lamprop-gui} is shown
below.\\

\begin{lstlisting}
import lamprop as la

t300 = la.Fiber(230000, 0.3, -0.41e-6, 1.76, 'T300-2')
epr04908 = la.Resin(2900, 0.36, 41.4e-6, 1.15, 'Epikote 04908')

L0 = la.Lamina(t300, epr04908, 100, 0, 0.50)
L90 = la.Lamina(t300, epr04908, 100, 90, 0.50)

cd0200 = la.Laminate('CD0200-2', (L0, L90, L90, L0))

print(la.text.engprop(cd0200))
\end{lstlisting}

This is probably the most flexible way to use it, since you could use Python
to generate large and complex laminates. Below, a symmetric and balanced
quasi-isotropic laminate with layers every 15\textdegree{} is generated. There
is no artificial limit to the amount of layers that can be defined. The author
has used laminates with up to 250 layers. Calculating the properties of that
laminate took approximately \SI{0.5}{s} on a machine with an Intel Core2 Q9300
running FreeBSD.

\begin{lstlisting}
import lamprop as la

t300 = la.Fiber(230000, 0.3, -0.41e-6, 1.76, 'T300-2')
epr04908 = la.Resin(2900, 0.36, 41.4e-6, 1.15, 'Epikote 04908')

layers = [la.Lamina(t300, epr04908, 100, a, 0.50) for a in range(-90, 95, 15)]
layers += layers[::-1]

qi = la.Laminate('quasi-isotropic', layers)

print(la.latex.out(qi, eng=True, mat=False))
print(la.latex.out(qi, eng=False, mat=True))
\end{lstlisting}

The resulting \LaTeX{} code basically produces \tref{tab:quasi-isotropic} and
\tref{tab:quasi-isotropic-mat}; the content was separated into two tables to
fit on the page.
\clearpage

\begin{table}[!htbp]
  \renewcommand{\arraystretch}{1.2}
  \caption{\label{tab:quasi-isotropic}Layers and engineering properties of quasi-isotropic}
  \centering\footnotesize{\rule{0pt}{10pt}
  \tiny calculated by lamprop 3.7\\[3pt]}
    \begin{tabular}[t]{rcrrl}
      \multicolumn{4}{c}{\small\textbf{Laminate stacking}}\\[0.1em]
      \toprule %% \usepackage{booktabs}
      Layer & Weight & Angle & vf & Fiber type\\
            & [g/m$^2$] & [$\circ$] & [\%]\\
      \midrule
      1 &  100 &   -90 & 50 & T300-2\\
      2 &  100 &   -75 & 50 & T300-2\\
      3 &  100 &   -60 & 50 & T300-2\\
      4 &  100 &   -45 & 50 & T300-2\\
      5 &  100 &   -30 & 50 & T300-2\\
      6 &  100 &   -15 & 50 & T300-2\\
      7 &  100 &     0 & 50 & T300-2\\
      8 &  100 &    15 & 50 & T300-2\\
      9 &  100 &    30 & 50 & T300-2\\
      10 &  100 &    45 & 50 & T300-2\\
      11 &  100 &    60 & 50 & T300-2\\
      12 &  100 &    75 & 50 & T300-2\\
      13 &  100 &    90 & 50 & T300-2\\
      14 &  100 &    90 & 50 & T300-2\\
      15 &  100 &    75 & 50 & T300-2\\
      16 &  100 &    60 & 50 & T300-2\\
      17 &  100 &    45 & 50 & T300-2\\
      18 &  100 &    30 & 50 & T300-2\\
      19 &  100 &    15 & 50 & T300-2\\
      20 &  100 &     0 & 50 & T300-2\\
      21 &  100 &   -15 & 50 & T300-2\\
      22 &  100 &   -30 & 50 & T300-2\\
      23 &  100 &   -45 & 50 & T300-2\\
      24 &  100 &   -60 & 50 & T300-2\\
      25 &  100 &   -75 & 50 & T300-2\\
      26 &  100 &   -90 & 50 & T300-2\\
      \bottomrule
    \end{tabular}\hspace{0.02\textwidth}
    \begin{tabular}[t]{rrl}
      \multicolumn{3}{c}{\small\textbf{Engineering properties}}\\[0.1em]
      \toprule
      Property & Value & Dimension\\
      \midrule
      $\mathrm{v_f}$ & 50 &\%\\
      $\mathrm{w_f}$ & 60.5 &\%\\
      thickness & 2.95 & mm\\
      density & 1.46 & g/cm$^3$\\
      weight & 4299 & g/m$^2$\\
      resin & 1699 & g/m$^2$\\
      \midrule
      $\mathrm{E_x}$ &    40924 & MPa\\
      $\mathrm{E_y}$ &    48752 & MPa\\
      $\mathrm{G_{xy}}$ &    15327 & MPa\\
      $\mathrm{\nu_{xy}}$ & 0.266215 &-\\
      $\mathrm{\nu_{yx}}$ & 0.317136 &-\\
      $\mathrm{\alpha_x}$ & 3.11462e-06 & K$^{-1}$\\
      $\mathrm{\alpha_y}$ & 2.12576e-06 & K$^{-1}$\\
      \bottomrule
    \end{tabular}
\end{table}

\begin{table}[!htbp]
  \renewcommand{\arraystretch}{1.2}
  \caption{\label{tab:quasi-isotropic-mat}Matrices of quasi-isotropic}
  \centering\footnotesize{\rule{0pt}{10pt}
  \tiny calculated by lamprop 3.7\\[3pt]}
  \resizebox{.8\hsize}{!}{
  \vbox{
    \vbox{\small\textbf{Stiffness (ABD) matrix}\\[-5mm]
      \tiny\[\left\{\begin{array}{c}
          N_x\\ N_y\\ N_{xy}\\ M_x\\ M_y\\ M_{xy}
        \end{array}\right\} = 
      \left|\begin{array}{cccccc}
           1.3206\times 10^{5} &  4.1882\times 10^{4} & 0 & 0 & 0 & 0\\
           4.1882\times 10^{4} &  1.5732\times 10^{5} & 0 & 0 & 0 & 0\\
          0 & 0 &  4.5286\times 10^{4} & 0 & 0 & 0\\
          0 & 0 & 0 &  7.9688\times 10^{4} &  2.8649\times 10^{4} & -1.8401\times 10^{4}\\
          0 & 0 & 0 &  2.8649\times 10^{4} &  1.3446\times 10^{5} & -2.9077\times 10^{4}\\
          0 & 0 & 0 & -1.8401\times 10^{4} & -2.9077\times 10^{4} &  3.1125\times 10^{4}\\
          \end{array}\right| \times
        \left\{\begin{array}{c}
            \epsilon_x\\[3pt] \epsilon_y\\[3pt] \gamma_{xy}\\[3pt]
            \kappa_x\\[3pt] \kappa_y\\[3pt] \kappa_{xy}
          \end{array}\right\}\]
    }
    \vbox{\small\textbf{Compliance (abd) matrix}\\[-5mm]
      \tiny\[\left\{\begin{array}{c}
            \epsilon_x\\[3pt] \epsilon_y\\[3pt] \gamma_{xy}\\[3pt]
            \kappa_x\\[3pt] \kappa_y\\[3pt] \kappa_{xy}
          \end{array}\right\} = \left|\begin{array}{cccccc}
           8.2705\times 10^{-6} & -2.2017\times 10^{-6} & 0 & 0 & 0 & 0\\
          -2.2017\times 10^{-6} &  6.9425\times 10^{-6} & 0 & 0 & 0 & 0\\
          0 & 0 &  2.2082\times 10^{-5} & 0 & 0 & 0\\
          0 & 0 & 0 &  1.4796\times 10^{-5} & -1.5802\times 10^{-6} &
          7.2712\times 10^{-6}\\
          0 & 0 & 0 & -1.5802\times 10^{-6} &  9.4889\times 10^{-6} &
          7.9304\times 10^{-6}\\
          0 & 0 & 0 &  7.2712\times 10^{-6} &  7.9304\times 10^{-6} &
          4.3835\times 10^{-5}\\
          \end{array}\right|\times
        \left\{\begin{array}{c}
            N_x\\ N_y\\ N_{xy}\\ M_x\\ M_y\\ M_{xy}
          \end{array}\right\}\]\\
    }
    }
    }
\end{table}

The stiffness or ABD matrix and compliance or abd matrix are shown with the
relevant strains and forces in \tref{tab:quasi-isotropic-mat}. Both are 6×6
matrices that can be divided into three 3×3 matrices; A, B and D or a, b and
d.

The expansions below reveal the symmetries in these matrices.

\[
    ABD = \left|\begin{array}{cccccc}
        A_{11} & A_{12} & A_{16} & B_{11} & B_{12} & B_{16}\\
        A_{12} & A_{22} & A_{26} & B_{12} & B_{22} & B_{26}\\
        A_{16} & A_{26} & A_{66} & B_{16} & B_{26} & B_{66}\\
        B_{11} & B_{12} & B_{16} & D_{11} & D_{12} & D_{16}\\
        B_{12} & B_{22} & B_{26} & D_{12} & D_{22} & D_{26}\\
        B_{16} & B_{26} & B_{66} & D_{16} & D_{26} & D_{66}\\
    \end{array}\right|\quad
    abd = \left|\begin{array}{cccccc}
        a_{11} & a_{12} & a_{16} & b_{11} & b_{12} & b_{16}\\
        a_{12} & a_{22} & a_{26} & b_{12} & b_{22} & b_{26}\\
        a_{16} & a_{26} & a_{66} & b_{16} & b_{26} & b_{66}\\
        b_{11} & b_{12} & b_{16} & d_{11} & d_{12} & d_{16}\\
        b_{12} & b_{22} & b_{26} & d_{12} & d_{22} & d_{26}\\
        b_{16} & b_{26} & b_{66} & d_{16} & d_{26} & d_{66}\\
    \end{array}\right|
\]

The units of the parts of the ABD and abd matrix are as follows (where $i$ and
$j$ are 1, 2 or 6):
$A_{ij}$ is in \si{N/mm}. $B_{ij}$ is in \si{N}.  $D_{ij}$ is in \si{N.mm}.
$a_{ij}$ is in \si{mm/N}.  $b_{ij}$ is in \si{1/N}.  $d_{ij}$ is in
\si{1/Nmm}.
% 
% \begin{description}
%     \item[$A_{ij}$] is in \si{N/mm}.
%     \item[$B_{ij}$] is in \si{N}.
%     \item[$D_{ij}$] is in \si{N.mm}.
%     \item[$a_{ij}$] is in \si{mm/N}.
%     \item[$b_{ij}$] is in \si{1/N}.
%     \item[$d_{ij}$] is in \si{1/Nmm}.
% \end{description}

The stress resultants $N$ are units of force per unit of length (\si{N/mm}).
Moment resultants $m$ are in units of torque per unit of length (\si{Nmm/mm}
= \si{N}). Both strains $\epsilon$ and $\kappa$ are dimensionless.


%%%%%%%%%%%%%%%%%%%% Eindmaterie %%%%%%%%%%%%%%%%%%%% {{{1
\setsecnumdepth{none}
%\include{appendices}
%\bibliography{lmref}
\chapter{Colofon}

This document has been typeset with the
\TeX\footnote{\url{http://nl.wikipedia.org/wiki/TeX}}
software, using the \LaTeX\footnote{\url{http://nl.wikipedia.org/wiki/LaTeX}}
macros and specifically the
\textsc{memoir}\footnote{%
    \url{http://www.ctan.org/tex-archive/macros/latex/contrib/memoir/}} style.

\end{document}
