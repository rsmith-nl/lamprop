% file: lamprop-manual.tex
% vim:fileencoding=utf-8:fdm=marker:ft=tex
%
% Copyright © 2018 R.F. Smith <rsmith@xs4all.nl>. All rights reserved.
% Created: 2018-05-13 23:48:04 +0200
% Last modified: 2018-05-15T22:41:25+0200

\newcommand{\twodigits}[1]{\ifnum#1<10 0\number#1\else\number#1\fi}
\newcommand{\TheDate}{\number\year-\twodigits\month-\twodigits\day}
\newcommand{\TheTitle}{Lamprop manual}

\documentclass[a4paper,landscape,oneside,11pt,twocolumn]{memoir}
\usepackage{fontspec}
\usepackage{siunitx}
\usepackage{ifthen}
\usepackage{enumitem}

% Set fonts for XeTeX {{{1
\setmainfont{Alegreya}[
    SmallCapsFont={Alegreya SC},
    SmallCapsFeatures={Letters=SmallCaps},
]
\setsansfont{TeX Gyre Heros}
\setmonofont{TeX Gyre Cursor}
\usepackage[italic]{mathastext}

% Hyperref moet als laatste geladen worden,
\usepackage[bookmarks,pdfborder={0 0 0}]{hyperref}
% met uitzondering van uitbreidingen erop.
\usepackage{memhfixc}

% Grootte van de pagina
\setlength{\trimtop}{0pt}
\setlength{\trimedge}{\stockwidth}
\addtolength{\trimedge}{-\paperwidth}
% Een A4 op zijn kant is 297 mm breed en 210 mm hoog.
% We halen 30 mm van de hoogte af, en 20 mm van de breedte.
\settypeblocksize{180mm}{277mm}{*}
% Volgens het log-bestand:
%   Text height and width: 514.20023pt by 788pt
%   Columnsep and columnseprule: 10pt and 0pt
% Dat betekent dat een kolom (788 - 10)/2 = 389pt = 137.2 mm is.
% Een foto op 300 dpi moet dan 389/72*300 = 1621 pixels breed zijn.
% Maar op de breedte van 1200 pixels passen er twee foto's in één
% kolom.

\setulmargins{*}{*}{1}
\setlrmargins{*}{*}{1}
\setheadfoot{\onelineskip}{1.5\onelineskip}
\setheaderspaces{*}{*}{1}
\checkandfixthelayout

% Instellingen van het siunitx pakket.
\sisetup{detect-all=true, mode=text, group-digits=true,
  input-decimal-markers={.,}, output-decimal-marker={.},
  exponent-product=\times, separate-uncertainty=true,
  load-configurations=abbreviations}

% voor enumitem
\setlist[itemize,1]{leftmargin=*}
\setlist[enumerate,1]{leftmargin=*}
\setlist[description,1]{leftmargin=*}
\setlist{noitemsep}

% Instellingen voor polyglossia
%\setmainlanguage[babelshorthands=true]{english}

% Definities van locale modificaties. MOET achter de invoeging van het ifthen
% pakket komen!
%\ifthenelse{\VCModified>0}%
%{\newcommand{\locmod}{\textcolor{red}{~(m)}}}{\newcommand{\locmod}{}}


% Document parameters {{{1
\title{\TheTitle}
\author{Roland F. Smith}
\date{\TheDate}

% Kop- en voetteksten
\makeevenhead{plain}{\thetitle}{}{}
\makeoddhead{plain}{\thetitle}{}{}
\makeheadrule{plain}{\textwidth}{\normalrulethickness}
\makefootrule{plain}{\textwidth}{\normalrulethickness}{0pt}
\makeevenfoot{plain}{\theauthor}{\thepage}{\TheDate}
\makeoddfoot{plain}{\theauthor}{\thepage}{\TheDate}

\makepagestyle{logboek}
\makeevenhead{logboek}{\thetitle}{}{\rightmark}
\makeoddhead{logboek}{\thetitle}{}{\rightmark}
\makeheadrule{logboek}{\textwidth}{\normalrulethickness}
\makefootrule{logboek}{\textwidth}{\normalrulethickness}{0pt}
\makeevenfoot{logboek}{\theauthor}{\thepage}{\TheDate}
\makeoddfoot{logboek}{\theauthor}{\thepage}{\TheDate}

% Ruimte voor nummering in lijsten
\cftsetindents{section}{1.5em}{3.0em}
\cftsetindents{figure}{1.5em}{3.0em}
\cftsetindents{table}{1.5em}{3.0em}

\sloppy
\makeindex

% Document info. {{{1
\special{pdf:docinfo <<
/Title (\TheTitle)
/Author (\theauthor)
/Subject (lamprop)
/Keywords (lamprop, manual, Roland Smith)
/CreationDate (D:20180108164304+0100)
>>}

%%%%%%%%%%%%%%%%%%%%%%%%% start van het document %%%%%%%%%%%%%%%%%%%%% {{{1
\begin{document}
% Specifiek voor MEMOIR
% Maak de lijsten minder open.
\tightlists
% Opmaak voor verklarende tekst bij figuren
\hangcaption
\captiontitlefont{\small}
\chapterstyle{section}
% Gebruik twee kolommen in TOC etc.
\doccoltocetc

% Geen inspringen, maar ruimte tussen paragrafen
%\setlength{\parskip}{\baselineskip}
\nonzeroparskip
\setlength{\parindent}{0pt}
\setbeforesecskip{5pt}
\setaftersecskip{1pt}
\setbeforesubsecskip{5pt}
\setaftersubsecskip{1pt}

\begin{titlingpage}
  \setlength{\parindent}{0pt} % Anders klopt uitlijning v.d. \rule's niet.
  \vspace*{\stretch{1}}
  \rule{\linewidth}{1mm}\vspace{5pt}
  \begin{flushright}
      {\Huge \TheTitle}\\[5mm]
      {\huge Roland F. Smith}
  \end{flushright}
  \rule{\linewidth}{1mm}
  \vspace*{\stretch{3}}
  \begin{center}
    \Large Eindhoven 2018
  \end{center}
\end{titlingpage}

% Om paginanummering gelijk te krijgen aan pagina's in PDF.
\setcounter{page}{2}
% Gebruik de standaard opmaak.
\pagestyle{logboek}

% Inhoudsopgave
\begin{KeepFromToc}
\tableofcontents
\end{KeepFromToc}
% Lijst met afbeeldingen.
%\listoffigures
% Lijst met tabellen
%\listoftables
% Index
%\printindex
\clearpage

% Alleen hoofdstuk- en sectie nummers.
\setcounter{secnumdepth}{1}

%%%%%%%%%%%%%%%%%%%% Inleiding %%%%%%%%%%%%%%%%%%%% {{{1
\chapter{Introduction}

The purpose of this program is to calculate some properties of
fiber-reinforced composite laminates. It calculates:
\begin{itemize}
    \item engineering properties like $E_x$, $E_y$ and $G_{xy}$
    \item thermal properties like $\alpha_x$ and $\alpha_x$
    \item physical properties like density and laminate thickness
    \item stiffness and compliance matrices (\textsc{abd}  and abd)
\end{itemize}


Although these properties are not very difficult to calculate, (the relevant
equations and formulas can be readily found in the available composite
literature) the calculation is time-consuming and error-prone when done by
hand.

This program can \emph{not} calculate the strength of composite laminates;
because there are many different failure modes, strengths of composite
laminates cannot readily be calculated from the strengths of the separate
materials that form the laminate. These strengths have to be determined from
tests.

The original version of this program was written in C, since implementing
it in a spreadsheet proved cumbersome, inflexible and even produced
incorrect results. The C version ran up to 1.3.x.

As an exercise in learning the language, the author ported the program to
the Python programming language. This proved to be a much cleaner, more
maintainable and shorter implementation.

In the meantime, the program was ported from python version 2 to python
version 3 and the core objects (in the \texttt{types.py} file) were made
immutable. Also the output method was made generic to enable output in
different formats, such as \LaTeX, \textsc{html} and \textsc{rtf}.

Additionally, the generally hard to obtain transverse fiber properties
were replaced with properties derived from the matrix.

\chapter{Building and installing the program}

\section{Requirements}

The main requirements are \texttt{python} (version 3.4 or later) and the
\texttt{numpy} library (version 1.6 or later). Currently the development is
done using \texttt{python} 3.6 and \texttt{numpy} 1.13.

For developers: You will need py.test\footnote{\url{https://docs.pytest.org/}}
to run the provided tests. Code checks are done using
pylama\footnote{\url{http://pylama.readthedocs.io/en/latest/}}. Both should be
invoked from the root directory of the repository.

There are basically two versions of this program; a console version (installed
as \texttt{lamprop}) primarily meant for \textsc{posix} operating systems and
a \textsc{gui} version (installed as \texttt{lamprop-gui}) primarily meant for
ms-windows.

You can try both versions without installing them first, with the following
invocations in a shell from the root directory of the repository.

Use \texttt{python3 -m lamprop.console -h} for the console version, and
\texttt{python3 -m lamprop.gui} for the \textsc{gui} version.


\section{Installation}

Run \texttt{python3 setup.py install}. This will install both the module and
the scripts that use it.




%%%%%%%%%%%%%%%%%%%% Eindmaterie %%%%%%%%%%%%%%%%%%%% {{{1
\setsecnumdepth{none}
%\include{appendices}
%\bibliography{lmref}
\chapter{Colofon}

This document has been typeset with the
\TeX\footnote{\url{http://nl.wikipedia.org/wiki/TeX}}
software, using the \LaTeX\footnote{\url{http://nl.wikipedia.org/wiki/LaTeX}}
macros and specifically the
\textsc{memoir}\footnote{%
    \url{http://www.ctan.org/tex-archive/macros/latex/contrib/memoir/}} style.

\end{document}
